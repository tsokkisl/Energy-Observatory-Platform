\documentclass[a4paper,12pt, oneside]{article}   	
	
\usepackage[utf8x]{inputenc}
\usepackage[english,greek]{babel}
\usepackage[margin=1.2in]{geometry}
\usepackage[parfill]{parskip}    	
\usepackage{microtype}
\usepackage{graphicx}
\usepackage{wrapfig}
\usepackage{enumitem}
\usepackage{fancyhdr}
\usepackage{amsmath,amsfonts,amssymb}
\usepackage{cite}
\usepackage{subcaption}
\usepackage{float}
\usepackage{titlesec}
\usepackage{index}
\makeindex

%SetFonts
\usepackage{lmodern}
%SetFonts
\usepackage[breaklinks=true, unicode=true]{hyperref}

\pagestyle{fancy}
\fancyhf{}
\renewcommand{\headrulewidth}{0pt}
\fancyhead[RE]{}
\fancyhead[RO]{}
\fancyfoot[RE, RO]{\thepage}


\titleformat{\section}
  {\normalfont\Large\bfseries}{\thesection}{1em}{}[{\titlerule[0.8pt]}]

\begin{document}

\begin{titlepage}
\begin{center}

\includegraphics[width=0.20\textwidth]{./img/NTUAlogo.jpg}~\\[0.1cm]
\textsc{ ΕΘΝΙΚΟ ΜΕΤΣΟΒΙΟ ΠΟΛΥΤΕΧΝΕΙΟ}\\[0.2cm]  
\textsc{ ΣΧΟΛΗ ΗΛΕΚΤΡΟΛΟΓΩΝ ΜΗΧΑΝΙΚΩΝ ΚΑΙ ΜΗΧΑΝΙΚΩΝ ΥΠΟΛΟΓΙΣΤΩΝ}\\[3cm] 


\textbf{\LARGE \textlatin{Stakeholders Requirements Specification}}\\[0.01cm]
\text{Σύμφωνα με το πρότυπο \textlatin{ISO/IEC/IEEE 29148:2011}}\\[2cm]

\textbf{\Large Εμπλεκόμενο μέρος: Εταιρίες παραγωγής\\ ~~~~~~~~~~~~~~~~~~~~~~ ηλεκτρικής ενέργειας}\\[2cm]

% Author and supervisor 
\textbf{Ομάδα:  \textlatin{WeAreBack}}\\
	Σταύρος Σταύρου \\
	Λένος Τσοκκής \\ 
	Κωνσταντίνα Φουντουραδάκη
\vfill

\end{center} 
\end{titlepage}

\thispagestyle{empty}
\pagenumbering{gobble}% Remove page numbers (and reset to 1)
\newpage

\thispagestyle{empty}
\pagenumbering{arabic}
\setcounter{page}{2}
\tableofcontents
\newpage


\section{Εισαγωγή}
\subsection{Ταυτότητα - Επιχειρησιακοί στόχοι}
Στόχος είναι η δημιουργία ενός συστήματος λογισμικού που εξασφαλίζει τη διαφάνεια στη λειτουργία της ευρωπαϊκής αγοράς ηλεκτρικής ενέργειας. Δίνει στα εμπλεκόμενα μέρη δυνατότητες διάθεσης, οπτικοποίησης και ανάλυσης \href{https://transparency.entsoe.eu}{ανοικτών δεδομένων.}

Μέσω του συστήματος αυτoύ οι παραγωγοί ηλεκτρικής ενέργειας μπορούν να έχουν καλύτερη εικόνα του ανταγωνισμού που υφίστανται, να ενημερώνονται για τους τύπους ηλεκτρικής ενέργειας που έχουν περισσότερη κατανάλωση ώστε να επενδύουν με μεγαλύτερη ασφάλεια το κεφάλαιό τους. Επιπλέον, μέσω του συστήματος έχουν πρόσβαση σε δεδομένα κατανάλωσης με ακρίβεια δεκαπεντάλεπτου, έτσι έχουν μια εκτίμηση της ζήτησης π.χ. κατά τη διάρκεια της ημέρας, της εβδομάδας, του έτους και μπορούν να διαχειριστούν καλύτερα το προσωπικό και το απόθεμά τους. Ακόμη, τους διατίθενται δεδομένα πρόβλεψης της κατανάλωσης ενέργειας για κάποια συγκεκριμένη μέρα, μήνα ή έτος μέσω των οποίων μπορούν να αποφασίσουν αν θα προβούν σε κάποια επέκταση ή θα υιοθετήσουν πολιτική λιτότητας. Τέλος, έχουν δυνατότητα να συγκρίνουν την πραγματική κατανάλωση ενέργειας που είχαν με την εκτίμηση που έλαβαν. Θα μπορούσαμε ως επέκταση του συστήματος μας να προσφέραμε στους παραγωγούς τη δυνατότητα απεριόριστης πρόσβασης (\textlatin{quota}) την ημέρα κατόπιν χρέωσης τους.

\subsection{Περίγραμμα επιχειρησιακών λειτουργιών}
Οι παραγωγοί ηλεκτρικής ενέργειας αφού συνδεθούν στο σύστημα μπορούν να πραγματοποιήσουν αναζήτηση για τιμές κατανάλωσης ενέργειας πραγματικές ή εκτιμώμενες, να τις συγκρίνουν καθώς και να αναζητήσουν τις τιμές παραγωγής ενέργειας. Για αυτές τις λειτουργίες μπορούν να εφαρμόσουν κάποια φίλτρα στα δεδομένα που θα τους επιστραφούν, όπως περιοχή, χρονική ανάλυση, χρονικο διάστημα και τύπο ενέργειας, αλλά και να  επιλέξουν το μορφότυπο τους.

\newpage
\section{Αναφορές-πηγές πληροφορίας}
\begingroup
%\renewcommand{\section}[2]{}
\renewcommand{\section}{\subsection}
\begin{thebibliography} {books}
\latintext
\bibitem{ISO2011} ISO/IEC 29148:2011 (IEEE Std 29148-2011), Systems and software engineering - Life cycle processes - Requirements engineering
\bibitem{ISO25060} ISO/IEC TR 25060:2010  Systems and software engineering - Systems and software product Quality Requirements and Evaluation - Common Industry Format for usability: General framework for usability-related information
\end{thebibliography}
\endgroup
\subsection*{Πηγές πληροφορίας}
\begingroup
\latintext
\sloppy
\begin{enumerate}
\begin{sloppypar}
   \item \href{https://transparency.entsoe.eu/content/static\_content/Static\%20content/knowledge\%20base/SFTP-Transparency\_Docs.html}{SFTP-Transparency Docs}
   \end{sloppypar}
   \item  \href{https://www.visual-paradigm.com}{https://www.visual-paradigm.com}
\end{enumerate}
\endgroup
\greektext


\section{Διαχειριστικές απαιτήσεις επιχειρησιακού περιβάλλοντος} 
\subsection{Επιχειρησιακό μοντέλο}
Η εφαρμογή θα λειτουργήσει και θα γίνει διαδεδομένη λόγω του ότι:
\begin{itemize}
  \item Η πλατφόρμα είναι αξιόπιστη καθώς εξασφαλίζει την ασφάλεια δεδομένων του χρήστη μέσω κρυπτογραφίας.
  \item Οι χρήστες έχουν πρόσβαση σε μια μεγάλη βάση δεδομένων κατανάλωσης ενέργειας.
  \item Τα δεδομένα που παρέχονται είναι έγκυρα, έγκαιρα και προστίθενται καταχωρήσεις ανά 15-λεπτο.
  \item Παρέχει διαφάνεια στην λειτουργία της αγοράς ηλεκτρικής ενέργειας.
\end{itemize}
Όλα τα παραπάνω θα βοηθήσουν ώστε οι επιχειρησιακοί στόχοι που έχουν οι εταιρίες παραγωγής να επιτευχθούν.

\subsection{Περιβάλλον διαχείρισης πληροφοριών}
Η σημερινή εικόνα για το περιβάλλον διαχείρισης πληροφοριών  στον τομέα της ηλεκτρικής ενέργειας περιλαμβάνει τους ακόλουθους οργανισμούς και πλατφόρμες. Η Ευρωπαϊκή ένωση παρέχει το σύστημα παρατήρησης \textlatin{EMOS} και ανά δύο χρόνια \href{https://ec.europa.eu/energy/en/data-analysis/market-analysis#gas-and-electricity-market-reports}{αναφορές} για την αγορά (τόσο τη βιομηχανική όσο και τα νοικοκυριά) ηλεκτρικής ενέργειας  που αναλύουν την εξέλιξη τιμών και όγκων κατανάλωσης καθώς και των αλληλεπιδράσεων μεταξύ των κρατών μελών της. Εταιρίες όπως η \href{https://www.enerdata.net}{\textlatin{Enerdata}}, \href{https://ec.europa.eu/eurostat/web/energy}{\textlatin{eurostat}}, \href{http://www.vaasaett.com}{\textlatin{vaasa ETT}} είναι μερικές από τις εταιρίες που προσφέρουν ανάλυση της αγοράς ενέργειας. 
\newpage


\section{Λειτουργικές απαιτήσεις επιχειρησιακού περιβάλλοντος}
\subsection{Επιχειρησιακές διαδικασίες}
Οι διαδικασίες της εφαρμογής που αναμένουν οι εταιρίες παραγωγής περιγράφονται ακολούθως.
\renewcommand{\theenumi}{\Roman{enumi}}
\renewcommand{\theenumii}{\arabic{enumii}}
\renewcommand{\theenumiii}{\alph{enumiii}}

\begin{enumerate}
   \item Πρόσβαση - Απεικόνιση δεδομένων
      \begin{enumerate}
     \item ανά περιοχή \textlatin{(Scope)}
     \item με συγκεκριμένη χρονική ανάλυση \textlatin{(Resolution)}
     \item για δοσμένη χρονική στιγμή \textlatin{(date)}
     \begin{enumerate}
  	\item συγκεκριμένη ημερομηνία
   	\item συγκεκριμένο μήνα και έτος	   	
	\item συγκεκριμένο έτος
     \end{enumerate}
     \item με μορφότυπο \textlatin{format}
     \begin{enumerate}
  	\item \textlatin{JSON}
   	\item \textlatin{.csv} αρχείο
     \end{enumerate}
   \end{enumerate}
   \item Σύγκριση δεδομένων
       \begin{enumerate}
     \item ανά περιοχή \textlatin{(Scope)}
     \item με συγκεκριμένη χρονική ανάλυση \textlatin{(Resolution)}
     \item για δοσμένη χρονική στιγμή \textlatin{(date)}
     \begin{enumerate}
  	\item συγκεκριμένη ημερομηνία
   	\item συγκεκριμένο μήνα και έτος	   	
	\item συγκεκριμένο έτος
     \end{enumerate}
     \item με μορφότυπο \textlatin{format}
     \begin{enumerate}
  	\item \textlatin{JSON}
   	\item \textlatin{.csv} αρχείο
     \end{enumerate}
   \end{enumerate}

   \item Σύνδεση στο σύστημα
    \begin{enumerate}
     \item σύνδεση με όσο το δυνατό λιγότερα πεδία προς συμπλήρωση, όπου \\ απαραίτητα είναι τα \textlatin{username} και  \textlatin{password}
     \item ανάκτηση κωδικού πρόσβασης
    \end{enumerate}
  \item Αποσύνδεση από το σύστημα
  \item Έλεγχος συνδεσιμότητας μεταξύ του χρήστη και της βάσης δεδομένων \textlatin{(HealthCheck)}
\end{enumerate}


\subsection{Δείκτες ποιότητας}
Δείκτες ποιότητας που σηματοδοτούν τη σωστή λειτουργία της εφαρμογής είναι ο μέσος αριθμός νέων χρηστών ανά ημέρα καθώς και το μεγάλο πλήθος εγγεγραμμένων χρηστών. Επιπλέον πολύ βασική ένδειξη ποιότητας είναι  η συχνότητα επισκέψεων των χρηστών. Ακόμη, δείκτης ποιότητας της πλατφόρμας είναι το πλήθος των δεδομένων με το οποίο ενημερώνεται αλλά και το εύρος περιοχών που καλύπτει. Τέλος καλό στατιστικό είναι και ο αριθμός κλήσεων ανά λειτουργική διαδικασία.

\renewcommand{\theenumi}{\arabic{enumi}}
\renewcommand{\theenumii}{\alph{enumii}}
\begin{enumerate}
   \item Ασφάλεια 
   \begin{enumerate}
     \item Τα στοιχεία των εγγεγραμμένων χρηστών είναι κρυπτογραφημένα.
     \item Χρήση πρωτοκόλλου \textlatin{HTTPS} για ασφαλή δικτυακή σύνδεση στην οποία τα δεδομένα ανταλλάσονται κρυπτογραφημένα.
   \end{enumerate}
   \item Περιεχόμενο της πλατφόρμας
   \begin{enumerate}
     \item Έγκαιρο, αφού καταχωρούνται καθημερινά στην πλατφόρμα νέα δεδομένα.
     \item Έγκυρα δεδομένα, τα οποία χρησιμοποιούνται και από τα κράτη μέλη της Ευρωπαϊκής Ένωσης.
     \item Ευρείας χωρικής κάλυψης
     \item Δεδομένα μονίμως καταχωρημένα, καθώς δεν διαγράφονται όταν απαρχαιωθούν.
   \end{enumerate}
\end{enumerate}


\section{Έκθεση απαιτήσεων χρηστών}
Ένας παραγωγός ενέργειας απαιτεί από την εφαρμογή δομημένη πρόσβαση στα δεδομένα, με δυνατότητα συγκρίσεων μεταξύ της δικιάς τους παραγωγής/κατανάλωσης ενέργειας με τις αντίστοιχες τιμές για άλλες εταιρίες ή τύπους ενέργειας. Ακόμη, οπτικοποίηση αυτών των συγκρίσεων και δεδομένων που αφορούν το προϊόν τους. Επιπλέον, επιθυμεί απεικόνιση διαγράμματος πίτας για τους τύπους ενέργειας και για τις εταιρίες παραγωγής ανά τύπο καθώς και ειδοποιήσεις όταν νέες εταιρίες εισάγονται στην αγορά, ώστε να έχει καλύτερη εποπτεία του ανταγωνισμού. Εύλογη απαίτηση είναι διαγράμματα που εμφανίζουν την κατανάλωση/παραγωγή ενέργειας ανά τύπο και τις διακυμάνσεις τους ώστε να παρακολουθούν τις τάσεις και εντοπίζουν επενδυτικές ευκαιρίες. Τέλος, επιθυμεί ασφαλή  αλληλεπίδραση, απεριόριστη πρόσβαση στο σύστημα, συνεχή ενημέρωση των δεδομένων και συμμόρφωση του συστήματος με τη διάταξη περί προστασίας προσωπικών δεδομένων \textlatin{(GPDR)}.


\section{Αρχές του προτεινόμενου συστήματος}
Οι παραγωγοί ηλεκτρικής ενέργειας επιθυμούν διεπαφή εύκολη στη χρήση, γρήγορη απάντηση στα αιτήματα τους και εύμορφη οπτικοποίηση των αποτελεσμάτων. Επιπλέον, βασική αρχή είναι να παρακολουθούν δεδομένα που αφορούν το προϊόν τους. Ακόμη, απαραίτητη αρχή συνιστά η προστασία από υποκλοπή των στοιχείων τους με χρήση κρυπτογραφίας. Τέλος, τα έγγραφα τεκμηρίωσης της εφαρμογής να είναι στα ελληνικά.


\section{Περιορισμοί στο πλαίσιο του έργου}
Περιορισμοί στο πλαίσιο του έργου αποτελούν:
\begin{itemize}
  \item Δυσκολία κατανόησης της χρήσης του \textlatin{CLI}
  \item Δυσκολία κατανόησης της χρήσης του \textlatin{SSH} για σύνδεση στο \textlatin{server}
  \item Απαιτείται σύνδεση στο διαδίκτυο για πρόσβαση σε τιμές που δεν είχαν ήδη ανακτηθεί ως \textlatin{.csv} αρχείο
  \item Η γλώσσα των δεδομένων και των διεπαφών είναι η αγγλική
\end{itemize}
\newpage

\section{Παράρτημα: ακρωνύμια και συντομογραφίες}
\textlatin{\textbf{EMOS} Environmental Management Overview Strategy\\
\textbf{REST API}  RESTful Application Programming Interface\\
\textbf{CLI} Command Line Interface\\
\textbf{JSON} JavaScript Object Notation\\
\textbf{HTTPS} Hypertext Transfer Protocol Secure\\
\textbf{SSH} Secure Shell \\
\textbf{GDPR} General Data Protection Regulation
}

\end{document}  